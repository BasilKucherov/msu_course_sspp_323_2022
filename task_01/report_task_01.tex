\documentclass[12pt,a4paper,oneside,final]{article}
\usepackage[utf8]{inputenc}
\usepackage[russian]{babel}
\usepackage{graphicx} % \includegraphics
\usepackage{indentfirst}

\oddsidemargin = 0cm
\topmargin = -1.5cm
\textwidth = 16cm
\textheight = 24cm
\parindent = 0.5cm

\newcommand\Section[1]{
  \refstepcounter{section}
  \section*{\raggedright
    \arabic{section}. #1}
  \addcontentsline{toc}{section}{%
    \arabic{section}. #1}
}

\newcommand\Subsection[1]{
  \refstepcounter{subsection}
  \subsection*{\raggedright
    \arabic{section}.\arabic{subsection}. #1
  }
  \addcontentsline{toc}{subsection}{%
    \arabic{section}.\arabic{subsection}. #1}
}

\newcommand\Subsubsection[1]{
  \refstepcounter{subsubsection}
  \subsubsection*{\raggedright
    \arabic{section}.\arabic{subsection}.\arabic{subsubsection}. #1
  }
  \addcontentsline{toc}{subsubsection}{%
    \arabic{section}.\arabic{subsection}.\arabic{subsubsection}. #1}
}

\sloppy

\title{Задание 1 \\
  Многопоточное вычисление числа $\pi$ с помощью библиотеки pthreads \\
  Отчёт}
\author{Кучеров\,В.Д.}
\date{2022}

\begin{document}

\maketitle

\Section{Постановка задачи}


Реализовать параллельный алгоритм с использованием интерфейса
POSIX Threads, вычисляющий число $\pi$, как интеграл:
\[
\int\limits_0^1 \frac{4}{1 + x^2} \ dx
\]
методом прямоугольников.

\Section{Формат коммандной строки}

\begin{verbatim}
./run <число отрезков разбиения> <число нитей>
\end{verbatim}

\Section{Спецификация системы}

\noindent
Процессор: Intel(R) Core(TM) i9-9880H CPU @ 2.30GHz

\noindent
Число вычислительных ядер: 8

\Section{Результаты выполнения}

Число отрезков: $N = 1000\,000\,000$

Для каждого числа нитей проводилось 3 эксперимента, в таблице представлено время каждого эксперимента,
усреднённое время и ускорение.

\begin{center}
  \begin{tabular}{|p{1.5 cm}|p{3 cm}|p{3 cm}|p{3 cm}|p{3 cm}|p{2 cm}|}
    \hline
    Число нитей $n$ & Эксперимент $1$ (с) & Эксперимент $2$ (с) & Эксперимент $3$ (с) & Среднее время работы (c) & Ускорение \\
    \hline
    $1$ & $3.518362$ & $3.610614$ & $3.620331$ & $3.583102$ & $1$ \\ 
    \hline
    $2$ & $1.752002$ & $1.758305$ & $1.866029$ & $1.792112$ & $1.999$ \\
    \hline
    $3$ & $1.204844$ & $1.183653$ & $1.248212$ & $1.212236$ & $2.955$ \\
    \hline
    $4$ & $0.904290$ & $0.906250$ & $0.967684$ & $0.926075$ & $3.869$ \\
    \hline
    $5$ & $0.749808$ & $0.772971$ & $0.796121$ & $0.772967$ & $4.635$ \\
    \hline
    $6$ & $0.644922$ & $0.621888$ & $0.648261$ & $0.638357$ & $5.613$\\
    \hline
    $7$ & $0.567938$ & $0.542898$ & $0.549812$ & $0.553549$ & $6.472$ \\
    \hline
    $8$ & $0.528188$ & $0.495816$ & $0.511634$ & $0.511879$ & $6.999$ \\ 
    \hline
  \end{tabular}
\end{center}

\end{document}
